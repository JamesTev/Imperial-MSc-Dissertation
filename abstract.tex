\noindent
The cost of existing brain-computer interface (BCI) technologies makes them inaccessible to the general public and largely prohibits their use on a mass scale. This project details the development of a novel, ultra low-cost BCI prototype that attempts to start changing this. This prototype, and its future extensions, is hoped to facilitate public engagement and education in the domain of EEG sensing and other neurotechnologies, as well as improving advocacy an encouraging further research funding.
\vspace{0.4cm}

\noindent
In particular, this study presents the development of real time decoding and communication of raw EEG signals acquired from an ultra low-cost proprietary EEG-based BCI device developed by the \href{https://www.imperial.ac.uk/next-generation-neural-interfaces/}{Next Generation Neural Interfaces} (NGNI) Lab at Imperial College. It forms part of a broader project being coordinated by the NGNI Lab that will be presented in an exhibition hosted by the Royal Society. The BCI prototype developed in this project will be worn by up to 100 different audience members at this exhibit and will serve to decode basic EEG signals concurrently and in real time in order to facilitate collaborative control in a multi-player game using only mental control.
\vspace{0.4cm}

\noindent
This project was designed under very challenging constraints, including: a budget of around 20 GBP (compared to the price of several hundred GBP for a typical BCI device), availability of only a single EEG channel and the limitation of dry surface electrodes. The steady state visual evoked potential (SSVEP) paradigm is used as the BCI control signal of choice owing to its ease of implementation, relatively high achievable SNR and the fact that it does not require user training. Based on their wide-spread used in the field of EEG signal processing, decoding algorithms based on canonical correlation analysis (CCA) such as Multi-set CCA (MsetCCA) and Generalised CCA (GCCA) are explored most closely. An MQTT client for publishing decoded data to a cloud service using Amazon Web Services (AWS) IoT Core is also implemented in firmware. Towards the objective of creating a \textit{fully mobile} BCI, all operations necessary for signal acquisition, processing, decoding and communication are capable of running on the device itself. This functionality is fully implemented in the firmware of the electronic hardware provided by the NGNI Lab based on the ESP32 SoC by Espressif Systems. All firmware is implemented in Micropython, the cross-architecture Python compiler and runtime for microcontrollers that runs on bare-metal.
\vspace{0.4cm}

\noindent
Subject to performing a short calibration sequence each time the BCI is put on, optimal results show that decoding accuracy of $95.56 \pm 3.74\%$ with an information transfer rate (ITR) of $102$ bits/min ($p=4$ calibration trials of $T=0.75$s each) can be achieved by the MsetCCA algorithm. With more modest calibration requirements ($p=2$ calibration trials of $T=1$s each), accuracy of $80.56 \pm 4.46\%$ with an ITR of $40$ bits/min can be achieved using the same algorithm. All decoding computation occurs on-device in real time. 
\vspace{0.4cm}

\noindent
Although the fact that data was only collected from a single test subject in this study is a significant limitation, the prototype produced nonetheless presents a very encouraging proof-of-concept that warrants further investigation and more stringent testing. Importantly, it has been developed with exclusively open source tools and all source code is freely available \href{https://github.com/JamesTev/EEG-decoding}{here}.

% \begin{itemize}
%     \item why have have I done this?\footnote{good question :)}
%     \item what does it do
%     \item what are the results
% \end{itemize}



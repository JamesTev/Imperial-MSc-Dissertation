\chapter{Conclusion and Future Work}
\label{chapter:conclusion}

\section{Conclusion}

\section{Future work}
While the system developed in this project performed remarkably well considering the constraints imposed on it, there are several areas for improvement. The following observations and suggestions are made for future extensions of this project. 
% Evaluate if objectives initially laid out were achieved
% - if not, why not
% - if they had been stated differently/more realistically, would the outcome have been different?
% - what would need to be done differently next time

\subsubsection{Multi-channel sensing}
As mentioned in the prior discussion, it is firmly believed that introducing additional EEG channels will significantly improve decoding performance. Furthermore, it is conceivable that the existing electronic hardware in this project could be adapted for multi-channel support by using a multi-channel analogue frontend (as is used on the OpenBCI Ganglion board, for example). The ESP32 SoC also has several available ADC inputs.

\subsubsection{Algorithm expansion}
While the algorithms selected in this project are some of the most popular and are close to the state-of-the-art at the time of writing, there are many other slight adjustments and extensions to these algorithms that warrant exploration. 

\subsubsection{Broader testing set}
As alluded to, the COVID-19 Pandemic made it difficult to gather data from different subjects during the course of this project. As a result, data was only collected on the author. This 
- mention how calibration averaging over many more people may make generalisation feasible


% \subsubsection{More robust testing}
% Mention idea of switching squares around to test robustness

\subsubsection{Improving comfort of electrodes}
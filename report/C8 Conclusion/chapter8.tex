\chapter{Conclusion and Future Work}
\label{chapter:conclusion}

\section{Conclusion}
Over the course of this project, an ultra low-cost, fully self-contained BCI prototype was developed. The research questions posed in Section \ref{section:research-questions} have been addressed:
\begin{enumerate}
    \item An ultra low-cost BCI device can be used for the purpose of decoding basic EEG signals such as SSVEPs in real time. Subject to performing a short calibration sequence each time the BCI is put on, decoding accuracy of $95.56 \pm 3.74\%$ with an ITR of $102$ bits/min ($p=4, \, T=0.75$s). With more modest calibration requirements ($p=2, \, T=1$s), accuracy of $80.56 \pm 4.46\%$ with an ITR of $40$ bits/min can be achieved.
    \item The device can achieve the aforementioned performance when performing \textit{all} operations necessary to do so on the device itself. No computational or other outsourcing is required.
\end{enumerate}
The results presented in Chapter \ref{chapter:results} demonstrate that the DSP system, coupled with the decoding algorithms used, proved very effective for SSVEP decoding. It can be concluded that, of the applicable algorithms explored, MsetCCA offers superior performance in terms of accuracy, ITR and robustness. While the results in this study are somewhat limited due to the fact that data was only collected from a single subject, they nonetheless present an encouraging proof-of-concept for further, more stringent investigation. 

At the time of writing, to the best of the author's knowledge, there are no fully mobile, self-contained BCI devices in the price range of around £20. Furthermore, unlike many existing research devices that use proprietary software or other tools, this prototype was developed using exclusively open source software. As a result, this project presents an exciting prospect for further research and development as it could become a valuable tool for education and public engagement. 

\section{Future work}
While the system developed in this project performed remarkably well considering the constraints imposed on it, there are several areas for improvement. The following observations and suggestions are made for future extensions of this project. 

\subsubsection{Multi-channel sensing}
As mentioned in the prior discussion, the limitation of only having one active EEG channel was significant. It is firmly believed that introducing addition channels would significantly improve decoding performance. Furthermore, it is conceivable that the existing electronic hardware in this project could be adapted for multi-channel support by using a multi-channel analogue frontend (as is used on the OpenBCI Ganglion board, for example). The ESP32 SoC also has several available ADC inputs.

\subsubsection{Algorithm expansion}
While the algorithms selected in this project are some of the most popular and are close to the state-of-the-art at the time of writing, there are many other slight adjustments and extensions to these algorithms that warrant exploration. For example, if multi-channel sensing is introduced, algorithms related to finding optimal spatial filters \textit{across channels} can be used. TRCA and its extensions (such as ensemble TRCA) are examples of such algorithms that have proved effective in the literature. 

\subsubsection{A more robust testing framework}
As alluded to, the COVID19 Pandemic made it difficult to gather data from different subjects during the course of this project. As a result, data was only collected from the author. An obvious future step would be to extend this experiment over many more test subjects of different ages with differing head morphologies and hair types. Having a broader test group could also be valuable for other reasons; subject to relevant data privacy considerations, the web logger infrastructure documented in Section \ref{fig:web-logger-c5} could allow anonymised EEG data to be collected remotely from a wide audience from the comfort of their homes. Optimising decoding algorithms using data from a wider variety of people would likely lead to a much greater chance of achieving effective generalisation across recording sessions with less need for calibration. 

Due to time constraints, testing procedures in this project were also fairly limited. More stringent validation tests would be helpful extensions. For example, data was collected under fairly idealised conditions where the subject did not move their head or eyes considerably. In future, the effect of some of the following disturbances on decoding performance could be investigated:
\begin{itemize}
    \item varying degrees of head and body movement
    \item eye movement and blinking
    \item modulation of focus on the stimulus and concentration in general
\end{itemize}
Another useful test would be to occasionally randomise the order of stimulus frequencies associated with the flashing squares in the SSVEP user interface. This could be used to verify that the decoding algorithms are indeed using valid EEG signals for frequency discrimination and not just other signal components correlated to the fixed position of the stimulus frequencies. 

Finally, experimenting with different SSVEP stimuli configurations and analysing the impact on decoding performance (if any) would be interesting. This could include varying the spatial distribution of stimuli as well as their colours and contrast relative to the background colour.

\subsubsection{Improved SSVEP squares user interface}
As depicted in Figure \ref{fig:ssvep-squares-c5}, the SSVEP squares user interface created for this project is very basic. A very useful extension would be to interpret signals published from BCI devices in order to provide user feedback of decoded signal results. Potentially, a means of visualising raw data could also be incorporated which would offer a far richer user experience. 

Another limitation of the current user interface is that the flicker frequencies of stimulus squares are controlled by vanilla Javascript. Consequently, there is likely to be fluctuations in the consistency of frequencies displayed on different devices and even on the same device during different computational loading conditions. In future, a more robust technology should be investigated to ensure that precise and consistent flicker frequencies are maintained. \href{https://www.khronos.org/webgl/}{WebGL} or related wrapper libraries such as \href{https://threejs.org/}{Three.js} could be potential options.

\subsubsection{Improved BCI headband ergonomics}
As mentioned in Section \ref{subsection:syst-limitations-discuss}, the current design of the BCI hardware becomes uncomfortable to wear, even after short periods of time. Before deploying this device in a public setting, work should be done to improve the ergonomics of the design to achieve greater comfort; ideally without any resulting compromises to the performance of the system.
% Chapter 4: Experimental Procedure 
\chapter{Apparatus and Experimental Procedure}
\label{chapter:experimental-procedure}

\graphicspath{ {report/Chapter4/assets/} } 

\section{Apparatus}

EEG systems are broadly comprised of the following elements:
\begin{itemize}
    \item electrodes: placed on the scalp of the subject to record raw electrical potentials. 
    \item signal processing elements: amplifiers and filters are typically employed before the signals are digitised
    \item analogue-to-digital converter (ADC): digitises measured signal for manipulation in a computer or microcontroller
    \item computer or microcontroller: facilitates data processing, computation and storage
\end{itemize}

\subsection{Embedded electronics}
Espressif ESP32 SoC: features that are relevant to this specific project include: 
\begin{itemize}
    \item dual-core, 240MHz CPU
    \item onboard FPU
    \item up to 600 DMIPS performance
    \item ultra low-power (ULP) co-processor
    \item 4 MiB SRAM
    \item integrated WiFi 802.11 b/g/n and BLE
\end{itemize}
Fortunately, the ESP32 SoC is extremely capable for its low cost (less than £5) and should be sufficient to handle on board computation for DSP and decoding provided that it is implemented efficiently. Its dual-core feature is also particularly useful as it will allow computation and networking to run concurrently. 
\subsection{EEG sensor hardware}

\section{Experimental Procedure}
\subsection{Data acquisition}
\subsection{Testing and verification}

\subsection{Demonstration idea}
The current idea is that users in the exhibition will be presented a mobile-friendly, lightweight web page with 2-4 stimulus squares or other images that flicker at frequencies $f_1, \dots, f_n$. Each stimulus will correspond to an action - such as `up' or `down' - that will control an avatar in the cooperative game mentioned above. The objective is to decode $f_1, \dots, f_n$ in order to interpret a user's desired action (i.e. discern which stimulus image they are focused on).
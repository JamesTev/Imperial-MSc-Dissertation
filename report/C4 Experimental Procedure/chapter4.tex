% Chapter 4: Experimental Procedure 
\chapter{Apparatus and Experimental Procedure}
\label{chapter:experimental-procedure}

\graphicspath{ {report/C4 Experimental Procedure/assets/} } 

\section{Apparatus}

In general, EEG systems are comprised of the following core elements \cite{teplan-eeg-measurement}, \cite{bci-survey-nicolas-alonso}:
\begin{itemize}
    \item electrodes: placed on the scalp of the subject to record raw electrical potentials. 
    \item signal processing elements: amplifiers and filters are typically employed before the signals are digitised
    \item analogue-to-digital converter (ADC): digitises measured signal for manipulation in a computer or microcontroller
    \item computer or microcontroller: facilitates data processing, computation and storage
\end{itemize}

\subsection{EEG hardware}
\subsubsection{OpenBCI Ganglion}
As the Imperial NGNI hardware prototype was still under development for a large part of this project, experimentation was initially done on a Ganglion bio-sensing kit made by \href{https://shop.openbci.com/products/ganglion-board?variant=13461804483}{OpenBCI} (OpenBCI, New York, USA). This kit was chosen due to the fact that it is relatively low-cost at 374.99 USD, open-source and has been scientifically validated as in \cite{autthasan-single-chan-ssvep}, \cite{peterson-bci-survey}. Furthermore, it is a relatively simple platform with sensing capabilities that are likely the most comparable to those anticipated for the NGNI prototype.

\begin{figure}
     \begin{subfigure}[c]{0.45\textwidth}
         \centering
         \includegraphics[width=\textwidth]{openbci}
         \caption{OpenBCI Ganglion board, electrodes and adjustable Velcro headband}
         \label{fig:openbci}
     \end{subfigure}
     \hfill
    \begin{subfigure}[c]{0.45\textwidth}
         \centering
         \includegraphics[width=\textwidth]{openbci-electrodes}
         \caption{Close-up view of the 4 active electrodes and two reference electrode clips (bottom)}
         \label{fig:openbci-electrodes}
     \end{subfigure}
        \caption[Images of the OpenBCI Ganglion bio-sensing device and electrodes]{Images of the OpenBCI Ganglion bio-sensing device and electrodes. The two black clips are the reference electrodes and the four active channels are received through the coloured wires.}
        \label{fig:openbci-subfigs}
\end{figure}
As depicted in Figure \ref{fig:openbci-subfigs}, the Ganglion board offers four active channels that. All four channels were using during experimentation as it was more convenient to exclude data from certain channels in post-processing than only measuring a subset of channels. Although wet electrodes can also be used with this kit, in accordance with the project constraints mentioned in Chapter \ref{chapter:introduction}, dry electrodes were used. As seen in Figure \ref{fig:openbci}, the four active electrodes have spiky nodules to increase surface pressure and thus improve contact quality with the scalp. For reference and comparison, some core features of the Ganglion board are provided below:
\begin{itemize}
    \item Microchip MCP3912 four channel 24-bit Delta-Sigma analogue frontend
    \item 200Hz sampling rate
    \item Simblee Bluetooth 4.0 module
\end{itemize}

\subsubsection{NGNI Prototype I}

As alluded to in Chapter \ref{chapter:introduction}, the hardware to be used in this project was supplied by the Imperial NGNI Lab. All hardware prototypes developed by the Lab were based on the Espressif ESP32; a low-cost, low-power SoC (system-on-chip) based on the Tensilica Xtensa LX6 microprocessor with integrated Wi-Fi and Bluetooth. Features of the ESP32 that are relevant to this project include \cite{esp32-digikey}:
\begin{itemize}
    \item dual-core, 240MHz CPU
    \item onboard FPU
    \item 12-bit SAR ADC
    \item 4x SPI, 2x I2C, 3x UART interfaces
    \item up to 600 DMIPS performance
    \item ultra low-power (ULP) co-processor
    \item 4 MiB SRAM
    \item integrated Wi-Fi 802.11 b/g/n and BLE
\end{itemize}
The ESP32 is extremely capable for its low price tag of around 3.6 USD \cite{esp32-digikey}. Its dual-core CPU is also particularly attractive as it could allow decoding-related computation and network communication to happen concurrently. 

\begin{figure}
     \centering
     \begin{subfigure}[b]{0.3\textwidth}
         \centering
         \includegraphics[width=\textwidth]{esp-top}
         \caption{Target board with ESP32 SoC and peripheral electronics}
         \label{fig:esp-hardware-soc}
     \end{subfigure}
     \hfill
     \begin{subfigure}[b]{0.3\textwidth}
         \centering
         \includegraphics[width=\textwidth]{programmer}
         \caption{Programmer board with serial-to-USB interface }
         \label{fig:esp-hardware-programmer}
     \end{subfigure}
     \hfill
    \begin{subfigure}[b]{0.3\textwidth}
         \centering
         \includegraphics[width=\textwidth]{esp-both}
         \caption{Bottom side of the programmer board (left) and target board}
         \label{fig:esp-hardware-both}
     \end{subfigure}
        \caption{Images of the final EEG hardware prototype developed in the Imperial NGNI Lab. This prototype includes the target board with ESP32 SoC, as well as a programmer board that is used to flash new firmware on to the microcontroller aboard the target board.}
        \label{fig:esp-hardware}
\end{figure}

\subsubsection{NGNI Prototype II}


\begin{figure}
     \begin{subfigure}[c]{0.48\textwidth}
         \centering
         \includegraphics[width=\textwidth]{final-headband}
         \caption{Programmer board with serial-to-USB interface }
         \label{fig:final-headband}
     \end{subfigure}
     \hfill
    \begin{subfigure}[c]{0.48\textwidth}
         \centering
         \includegraphics[width=\textwidth]{final-headband-elec}
         \caption{Bottom side of the programmer board (left) and target board}
         \label{fig:final-headband-electrodes}
     \end{subfigure}
        \caption{Images of the final EEG hardware prototype developed in the Imperial NGNI Lab. This prototype includes the target board with ESP32 SoC, as well as a programmer board that is used to flash new firmware on to the microcontroller aboard the target board.}
        \label{fig:final-headband-subfigs}
\end{figure}

\section{Experimental Procedure}
\subsection{Data acquisition}
\subsection{Testing and verification}

\subsection{Demonstration idea}
The current idea is that users in the exhibition will be presented a mobile-friendly, lightweight web page with 2-4 stimulus squares or other images that flicker at frequencies $f_1, \dots, f_n$. Each stimulus will correspond to an action - such as `up' or `down' - that will control an avatar in the cooperative game mentioned above. The objective is to decode $f_1, \dots, f_n$ in order to interpret a user's desired action (i.e. discern which stimulus image they are focused on).
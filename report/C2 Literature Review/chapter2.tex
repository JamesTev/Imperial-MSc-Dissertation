% Chapter 2: Lit Review
\chapter{Literature Review}

\graphicspath{ {report/Chapter2/assets/} } 

% 2.1: EEG and neuroanatomy
\section{Basic Electrophysiology and Electroencephalography}

\subsection{Electrophysiology and basic neuroanatomy}

\subsection{Electroencephalography Paradigms}

\subsection{Steady-state visual evoked potentials (SSVEP)}

% 2.2: EEG Mechanics
\section{EEG Signal Acquisition and Processing}
Discuss the state-of-the-art in EEG signal acquisition and processing techniques. Discuss the mechanics of existing, commonly used devices both in industry and academia.
\subsection{Experimental parameters in SSVEP decoding and their implications}
\subsubsection{Number of EEG channels used}
\subsubsection{Characterisation of visual stimuli}
Discussion on parameters that can be changed in SSVEP experiments and the implications thereof:
\begin{itemize}
    \item Number of visual stimuli
    \item Contrast and colour
    \item Relative size and positioning of stimuli
    \item Frequencies
\end{itemize}

\subsection{Practical considerations}
Potentially list any practical findings from the literature that would influence our design choices. For example, refresh rate of screens and the impact on consistency of flickering stimuli. Or perhaps just the hardware that was used in experiments in the literature and the importance of that (if any)

\section{Computational Approaches for SSVEP Decoding}
\subsection{Power spectral density and frequency domain}
\subsection{Statistical}
\subsection{Data-driven and machine learning}
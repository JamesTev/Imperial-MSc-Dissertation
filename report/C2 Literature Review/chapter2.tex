% Chapter 2: Lit Review
\chapter{Literature Review}
\label{chapter:lit-review}

\graphicspath{ {report/C2 Literature Review/assets/} } 

% 2.1: EEG and neuroanatomy
\section{Basic neurophysiology}

\subsection{Electrophysiology}
EEG relies on electrophysiological activity generated by electro-chemical neurotransmitters that exchange signals between neurons in the brain \cite{bci-survey-nicolas-alonso}. Roughly speaking, when a neuron is excited by afferent action potentials (rapid changes in cell membrane potentials), postsynaptic potentials (EPSPs) are generated in its apical dendrites \cite{baillet-em-brain-mapping} (neural branches). This induces a potential difference (dipole) which results in a small flow of current from the nonexcited soma membrane to the apical dendrites where the EPSPs are present. \cite{baillet-em-brain-mapping}. This process causes both intracellular current flow within the neuron, as well as extracellular current flow. These current flows are termed \textit{primary} and \textit{secondary} or \textit{return} currents respectively. Both such currents contribute to electrical potentials measured on the scalp, however, it is believed that the source of most measurable EEG potentials arises from large collections of cortical pyramidal neurons arranged in macro-assemblies with dendrites orientated perpendicularly to the local cortical surface \cite{baillet-em-brain-mapping}, \cite{teplan-eeg-measurement}. The specific spatial positioning and simultaneous activation of these large clusters of neurons is believed to generate signals that can be measured on the scalp. Specifically, \cite{baillet-em-brain-mapping} suggests that these signals most likely arise from EPSPs in these macro-assemblies and less so due to rapidly firing action potentials that travel along the axons of excited neurons.

\subsection{Functional neuroimaging}
Various techniques exist for measuring and translating brain activity into electrical signals. These techniques are typically grouped into electrophysiological and hemodynamic. Hemodynamic techniques such as functional magnetic resonance (fMRI) and near infrared (NIR) spectroscopy measure brain activity \textit{indirectly }by tracking relative concentrations of oxyhemoglobin. The blood releases glucose to \textit{active} neurons at a greater rate than inactive ones which causes an increase in oxyhemoglobin in these active areas \cite{bci-survey-nicolas-alonso}, thereby providing a measurable proxy for cerebral activity. While hemodynamic approaches offer superior spatial resolution, they typically offer far lower temporal resolution and require complex, expensive equipment. 

EEG, on the other hand, is a electrophysiological technique that relies on the direct measurement of electrical signals generated by neural cell assemblies. Owing to the fact that it offers high temporal resolution\footnote{typically in the order of tens of milliseconds}, far lower cost than most other techniques, high portability and safety, EEG has become the most widely used neuroimaging modality \cite{bci-survey-nicolas-alonso}. It is these factors that make EEG the most appropriate modality for this project.

\section{Electroencephalography}

\subsection{Invasive vs non-invasive techniques}

The ability to reliably detect EEG signals is made challenging by the fact that neuron potentials must pass through the skull and scalp and will unavoidably be measured in conjunction with background noise and other undesirable artefacts such as electromyography (EMG) signals. Invasive methods that require surgical implantation of electronic devices are sometimes used in order to circumvent some of these challenges. Electrocorticography or intercranial EEG is an example of an invasive method; an electrode array is placed directly on the exposed surface of the brain to measure cerebral activity. 

While invasive methods offer superior signal resolution and quality, they are clearly not suitable for this project. Non-invasive BCIs do not require any surgical intervention and only involve the placement of electrodes on the scalp of the subject. Despite the aforementioned challenges of measuring signals of poorer quality, this has the advantage of convenience, cost effectiveness, safety and minimal invasiveness.

\subsection{The nature of EEG signals}
EEG signals most commonly range in amplitude from 0.5 to $100\mu$V peak-to-peak \cite{teplan-eeg-measurement} (in the case of a healthy brain). For reference, this is roughly 100 times lower than the typical amplitude of ECG signals. It is widely believed that EEG signals show varying energy in a few distinct frequency bands depending on mental state and cognitive function of a subject \cite{baillet-em-brain-mapping}, \cite{varnavas-phd}. These frequency bands are summarised below \cite{varnavas-phd}, \cite{bci-survey-nicolas-alonso}:

\begin{table}[!htb]
\centering
\begin{tabular}{@{}ll@{}}
\toprule
\textbf{Frequency band} &
  \textbf{Physiological association} \\ \midrule
\textit{delta} (0.5 - 4Hz) &
  \begin{tabular}[c]{@{}l@{}}Usually only detected in a state of deep sleep. Excessive signal energy in the delta band \\ while awake may suggest neurological disease.\end{tabular} \\
\textit{theta} (4 - 8Hz) &
  \begin{tabular}[c]{@{}l@{}}Cognitive tasks involving association, awareness and meditation. Usually low energy\\ in this band while subject is awake.\end{tabular} \\
\textit{alpha} (8 - 13Hz) &
  \begin{tabular}[c]{@{}l@{}}Typically measured in the occipital region of the brain. Primarily related to visual processing \\ but also memory processes. Induced by closing of the eyes and relaxing and attenuated when \\ eyes are open or by thinking or mental calculation.\end{tabular} \\
\textit{beta} (12 - 30Hz) &
  \begin{tabular}[c]{@{}l@{}}A variety of mental processes such as mathematical computation, planning, high \\ level processing\end{tabular} \\ \bottomrule
\end{tabular}
\caption{Summary of commonly designated EEG signal frequency bands and their broad physiological associations}
\label{tab:c2-freq-bands}
\end{table}

It is worth noting that different areas of the brain may produce signals with different energy compositions across these frequency bands. Consequently, the particular placement of electrodes in measuring localised signals of interest is an important consideration.

\subsection{Electrode choice and placement}
An EEG is signal is most commonly measured as the potential difference over time between active and reference electrodes. A third ground electrode is used to measure differential voltage between the active and reference electrodes and thus, a minimal setup requires at least 3 electrodes in total \cite{bci-survey-nicolas-alonso}. Electrodes be used in conjunction with conductive media such as conductive gel or without and would be termed wet or dry electrodes in these two cases respectively. 

Electrodes are most commonly arranged on the scalp according to the International 10-20 system standardised by the American Electroencephalographic Society. An overview of this system is presented in Figure \ref{fig:10-20-positions}.

\begin{figure}[h]
    \centering
    \includegraphics[width=0.8\textwidth]{10-20-electrode.png}
    \caption[Electrode positions according to the 10-20 system]{Diagram illustrating typical electrode placements using the international 10-20 system. A represents the ear lobe, C the central region, P the parietal, F the frontal and O the occipital. The \textit{naison} and \textit{inion} are used as reference locations.}
    \label{fig:10-20-positions}
\end{figure}

\subsection{BCI control signals}
The core role of a BCI is to interpret the intentions of a subject by making sense of their brain signals. These signals are comprised of a superposition of many different neuronal potentials associated with various mental tasks, most of which are not yet understood or clearly identifiable. However, there are some mental processes that have proved to correspond to identifiable signals that can be decoded by BCIs. These signals are either produced predictably in response to a particular external stimulus, or can be be modulated at will by a subject with suitable conditioning \cite{bci-survey-nicolas-alonso}. Some of the most popular BCI control signals are discussed below.

\subsubsection{Visual evoked potentials}
Visual evoked potentials (VEPs) are modulations in the activity of the brain's visual cortex in response to a visual stimulus \cite{bromm-veps}. VEPs can be characterised by the nature of the visual stimulus, namely: the flicker or reversal frequency of the, the morphology of the stimulus itself and by proportion of the visual field occupied by the stimulus \cite{bci-survey-nicolas-alonso}. Most commonly, frequency is modulated between different visual stimuli in order to encode different control targets. Transient VEPs, as the name suggests, are short-term potentials that occur in response to visual stimuli below 6Hz. Conversely, steady state VEPs (SSVEPs) are produced at frequencies above 6Hz and are characterised by sinusoidal signals with a fundamental frequency matching that of the visual stimulus \cite{Xie2016}. VEPs have the advantage of a relatively high information transfer rate (ITR) and do not require any training from the subject as they are elicited involuntarily. 

\subsubsection{Slow cortical potentials}
Slow cortical potentials (SCPs) are slow voltage shifts below 1Hz that correspond to changes in the level of cortical activity \cite{bci-survey-nicolas-alonso}. Although these signals can be self-regulated in order to control external devices through a BCI, reliable operation often requires training and is dependent on numerous external factors such as a subject's psychological state, motivation and social context \cite{bci-survey-nicolas-alonso}. Furthermore, as SCPs occur over several seconds, maximum ITRs attainable are low.

\subsubsection{Event related potentials}
Event related potentials (ERPs), such as the P300 evoked potential, are positive signal spikes generated in response to infrequent or unexpected auditory, visual or somatosensory stimuli \cite{bci-survey-nicolas-alonso}. `P300' is derived from the fact that these potentials are typically evoked around 300ms after observing an infrequent stimulus after a sequence of common or expected ones. P300-based BCIs are only capable of very low ITRs since considerable number of non-target stimuli need to be presented before the infrequent stimulus in order to preserve its novelty. Furthermore, magnitude of P300 responses may decrease with time as the subject begins to anticipate responses. 

\subsubsection{Sensorimotor rhythms}
Sensorimotor rhythms are modulations in cerebral activity associated with motor tasks, even when overt motor action is not performed. BCI control can be achieved through sensorimotor signals as subjects can be trained to generate modulations voluntarily through mental rehearsal of motor actions. For example, a subject may imagine clenching their fists. However, obtaining reliable signals through self-control proves difficult in practice as patients often imagine \textit{visual images } of related movements which does not produce sufficiently similar cerebral activity to the image of performing the actual activity itself \cite{bci-survey-nicolas-alonso}. As such, reliable BCI control usually requires special training with an emphasis on kinesthetic experiences.

\begin{table}[]
\centering
\begin{tabular}{@{}llll@{}}
\toprule
\textbf{Paradigm} &
  \textbf{Physiological phenomena} &
  \textbf{Training required} &
  \textbf{ITR (bits/min)} \\ \midrule
VEP          & \begin{tabular}[c]{@{}l@{}}Signal modulations in the visual cortex in response\\ to a flickering visual stimulus\end{tabular}     & No  & \textbf{60-200} \\
SCP          & \begin{tabular}[c]{@{}l@{}}Slow shifts in cortical potentials due to modulation\\ of cortical activity/concentration\end{tabular} & Yes & 5-12   \\
ERP &
  \begin{tabular}[c]{@{}l@{}}Abrupt signal modulations in response to infrequent\\ or unexpected stimuli\end{tabular} &
  No &
  20-25 \\
Sensorimotor & \begin{tabular}[c]{@{}l@{}}Modulations in signals from the motor cortex \\ synchronised to imagined motor actions\end{tabular}    & Yes & 3-35   \\ \bottomrule
\end{tabular}
\caption[A summary of commonly used BCI control signal paradigms]{A summary of commonly used BCI control signal paradigms. The information transfer rate (ITR) values provided are typical but may vary depending on the type of data acquisition and decoding systems used.}
\label{tab:c2-bci-control-signals}
\end{table}

\section{Steady-state visual evoked potentials (SSVEP)}
Several studies suggest that steady-state visual evoked potentials  \cite{Fernandez-Fraga2016}, \cite{Kanoga2020}, \cite{Acampora2021} (SSVEPs) offer significant potential for EEG decoding tasks similar to those in this project due to the high information transfer rate (ITR), non-invasiveness and relatively high SNR that can be achieved using basic BCI devices \cite{Zhu2021}. Moreover, SSVEP amplitudes change as a function of stimulus intensity (luminance and contrast) \cite{autthasan-single-chan-ssvep} which can easily be controlled. Considering these factors and the fact that SSVEP-based BCIs require little to no user training or prior BCI experience, it is clear that this is the most suitable EEG paradigm for this project. 

\subsection{Evoking and measuring SSVEPs}

Broadly speaking, in order to evoke SSVEPs, flickering visual stimuli with frequencies of around 7-15Hz are commonly used \cite{Acampora2021}, \cite{Chen2017}, \cite{duart-comparing-ssvep-stimuli}, however, Xie et al demonstrated successful decoding with 27.8Hz \cite{Xie2016}. As in \cite{Acampora2021}, \cite{Chen2017}, \cite{autthasan-single-chan-ssvep}, visual stimuli are usually presented using a computer monitor with an LCD display or discrete LEDs. SSVEPs are measured in the occipital and/or parietal regions \cite{Fernandez-Fraga2016} for proximity to the visual cortex. Using the international 10-20 electrode electrode scheme, this corresponds to the parieto-occipital ($\text{PO}_x$) and occipital ($\text{O}_x$)  electrodes.

The aforementioned studies do not provide a unanimous set of parameters for the optimal SSVEP stimulus. There is substantial variation in the frequency, colour and source of stimulus used across these studies and indeed, many other studies in the literature \cite{duart-comparing-ssvep-stimuli}. Duart et al in \cite{duart-comparing-ssvep-stimuli} sought to investigate these stimulus design parameters more closely and compare their impact on accuracy of frequency discrimination and SNR (signal-to-noise ratio) in SSVEP-based BCIs. Specifically, the following parameters were tested:
\begin{itemize}
    \item \textbf{frequency}: low (5Hz), middle (12Hz) and high (30Hz) frequencies were tested.
    \item \textbf{colour}: red, white and green squares were experimented with. 
    \item \textbf{attention}: a measure of attention was tested using to determine correlation with evoked responses.
\end{itemize}
Duart et al found that the middle frequency of 12Hz produced maximal SNR, followed by the low 5Hz frequency. Furthermore, red and green stimuli produced responses with maximum SNR at 5Hz while red and white were optimal at 12Hz. No difference in SNR was observed between colours at 30Hz. Despite red light proving optimal in other studies such as \cite{chu-ssvep-colours}, Duart et al note that red light may produce increased risk of inducing epileptic seizures and should thus be avoided where possible. Moreover, Zhu et al noted that lower (flicker) frequencies and light colours of longer wave length tend to produce greater visual fatigue which consequently degrades SSVEP responses over time \cite{zhu2010survey}. Finally, Duart et al measured attention of subjects undergoing SSVEP trials through the Conner's Continuous Performance Task version 2 (CPT-II) which measures reaction times, omission errors and commission errors \cite{duart-comparing-ssvep-stimuli}. They found that this measure of attention showed a significant correlation to the SNR of evoked SSVEP signals at the low frequency of 5Hz, but less so with higher frequencies. The authors hypothesised that this may be due to the fact that greater concentration is required at lower frequencies that cause greater fatigue \cite{duart-comparing-ssvep-stimuli}.

Taking into account these findings, it would appear that using stimulus frequencies of around 12Hz would be a suitable choice. Furthermore, so as to avoid using red light, white would be a good choice in medium frequency range and green would be suitable for lower frequencies around 5Hz. Using frequencies around 12Hz has the added advantage of causing less visual fatigue and theoretically, should require less active concentration from subjects in order to achieve adequate SNR.

% 2.2: EEG Mechanics
\section{EEG Signal Acquisition and Processing}
Discuss the state-of-the-art in EEG signal acquisition and processing techniques. Discuss the mechanics of existing, commonly used devices both in industry and academia.


\subsection{Practical considerations}
Potentially list any practical findings from the literature that would influence our design choices. For example, refresh rate of screens and the impact on consistency of flickering stimuli. Or perhaps just the hardware that was used in experiments in the literature and the importance of that (if any)

\section{Computational Approaches for SSVEP Decoding}
\subsection{Power spectral density and frequency domain}
\subsection{Statistical}

\subsubsection{Canonical correlation analysis (CCA)}
CCA is a standard multivariate statistical technique for analysing multiple variables measured on a set of observations. In particular, variables are partitioned into two sets or \textit{views} of the data \cite{cca-tutorial}. Effectively, CCA is a multivariate extension of ordinary correlation. Given variable partitions $\mX \in \R^{m\times p}$, $\mY \in \R^{m\times q}$ each with $m$ observations, CCA seeks to find a linear combination of $\mX$, $\mY$ that maximises the correlation between their images $\vz_X = \mX\vw_X$ and $\vz_Y = \mY\vw_Y$. These images are known as the \textit{canonical variables} and $\vw_X$ and $\vw_Y$ are the \textit{canonical weight vectors} that effectively act as spatial filters across signal channels.

With CCA for SSVEP frequency recognition tasks, $\mX \in \R^{N_s\times N_c}$ is the set of \textit{measured} EEG signals from $N_c$ channels over $N_s$ observations. The measured signals in $\mX$ must be compared with each frequency $f_k \in \mathcal{F}$, a finite set of candidate stimulus frequencies. CCA can be used to compute weighted correlations between the measured signals and each candidate frequency $f_k$ by constructing a sinusoidal reference signal set $\mY_k \in \R^{N_s \times 2N_h}$ as follows: 
\begin{equation}
\mY_{k}=\left(\begin{array}{c}
\sin \left(2 \pi f_{k} t\right) \\
\cos \left(2 \pi f_{k} t\right) \\
\sin \left(4 \pi f_{k} t\right) \\
\cos \left(4 \pi f_{k} t\right) \\
\vdots \\
\sin \left(2 \pi N_h f_{k} t\right) \\
\cos \left(2 \pi N_h f_{k} t\right)
\end{array}\right), \quad t=\frac{1}{f_s}, \frac{2}{f_s}, \ldots, \frac{N_s}{f_s}
\end{equation}
where $f_s$ is the sampling frequency and $N_h$ is the number of harmonics in the reference set, a parameter to be chosen. The scalp and other interacting layers between the brain and surface electrodes are both resistive and capacitive, producing low-pass filter dynamics \cite{baillet-em-brain-mapping}, \cite{lin-cca-2006}. Therefore, it is rare that more than $N_h=5$ harmonics are selected in the reference signal set \cite{lin-cca-2006}. The final task of frequency recognition is performed by selecting the frequency $f^*$
\begin{equation}
    f^* = \argmax_{f_k} \; \rho_k, \quad \forall \, f_k \in \mathcal{F}
    \label{eq:cca-freq-discrimination}
\end{equation}
corresponding to the candidate frequency $f_k$ which maximises the canonical correlation $\rho_k$ between $\mX$ and the the reference set $\mY_k$. While standard CCA is a good starting point for statistical SSVEP decoding as it forms the basis of many similar algorithms, there are several extensions which have been shown to improve signal recognition performance \cite{zhang-mset-cca}, \cite{sun-gcca}, \cite{miao-hybrid-cca}. The most promising of these are explored below.

\subsubsection{Multiset canonical correlation analysis (MsetCCA)}
MsetCCA is one extension of standard CCA that takes into account historical data instead of performing inference purely on new observations. Zhang et al propose that this is one of the reasons that standard CCA performs poorly on short time windows; it effectively over fits to localised dynamics \cite{zhang-mset-cca}. Furthermore, the authors suggest that exclusively using the pre-constructed sinusoidal reference set is not optimal since this artificial reference does not exclude other features from real EEG data \cite{zhang-mset-cca}. To circumvent this, MsetCCA seeks to optimise the reference signals used in the CCA algorithm by learning multiple linear transforms to maximise overall correlation between canonical variables over \textit{many} sets of EEG data at each candidate frequency $f_k \in \mathcal{F}$ \cite{zhang-mset-cca}. This optimisation effectively finds optimal joint spatial filters $\vw_1, \dots, \vw_{N_t}$ (over $N_t$ trials) using only historical observations (`training' data). The authors claim that MsetCCA outperforms similar techniques, especially in cases with few channels and short time windows. 

\subsection{Data-driven and machine learning}
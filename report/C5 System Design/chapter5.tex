%% Chapter 5: System design
\chapter{System Design}
\label{chapter:system-design}

\graphicspath{ {report/C5 System Design/assets/} } 

\section{Design of the SSVEP Stimuli}
\label{subsection:ssvep-stimuli}
A key consideration in the design of the SSVEP stimuli is the stimulus frequencies $f_1, \dots, f_n$ to use. As mentioned in Section \ref{subsection:time-frequency-considerations-c2}, similar studies typically use stimulus frequencies between 7Hz and 12Hz for this task. Furthermore, as detailed in Section \ref{subsection:CCA-c2}, CCA and CCA-based decoding algorithms involve a harmonic reference signal set. An important consideration is the \textit{number of harmonics} to include in these reference sets. In order to allow for a fundamental stimulus frequency range of 7Hz to 12Hz, it is only feasible to include one harmonic ($N_h=1$) in the reference set since the second harmonic of a 12Hz stimulus signal would occur at 36Hz which would likely be attenuated somewhat by the analogue low-pass filter (corner frequency of 37.4Hz) and low-pass filter dynamics of the adjustable output amplifier (corner frequency of 36.35Hz) shown in Figure \ref{fig:analogue-system-c4}. Furthermore, including the second harmonic for each stimulus frequency would have significant memory and computation implications. It is for these reasons that only a single harmonic was used in this system (where applicable).

\subsection{SSVEP stimulus interface}

Naturally, an important part of the SSVEP-based BCI system is the SSVEP stimuli which are to be presented to individuals participating in the exhibition (or in general, anyone interacting with the system thereafter). Taking into account design parameters such as stimulus colour, position and contrast mentioned in the literature cited in Section \ref{subsection:evoking-measuring-ssveps}, a basic user interface (UI) was designed with a series of flashing squares as depicted in Figure \ref{fig:ssvep-squares-c5}. This UI was implemented\footnote{code implementation available \href{https://github.com/JamesTev/EEG-decoding/blob/master/ui/ssvep_squares.html}{here}.} as a lightweight HTML page with basic CSS styling and Javascript to handle animation (flickering of the squares). This decision was made to allow for the simplest and most convenient deployment across any device capable of displaying a web page (mobile or otherwise). The number of blocks being displayed and their labels are subject to adjustment in accordance with other elements of the demonstration beyond the scope of this project.

\begin{figure}[!htb]
    \centering
    \includegraphics[width=0.6\textwidth]{ssvep-squares}
    \caption{Screen capture of the user interface for displaying SSVEP stimuli. The blocks can be independently set to any flicker frequency of interest.}
    \label{fig:ssvep-squares-c5}
\end{figure}

Note that, external factors related to this project may only require some subset of the stimuli shown in Figure \ref{fig:ssvep-squares-c5}; for example, the square corresponding to `down' may be omitted if only the other three actions are required in the game/simulation. 

\section{Design of the Digital System}
\begin{figure}[!htb]
    \centering
    \includegraphics[width=\textwidth]{digital-system-overview}
    \caption{Overview diagram of the core components that comprise the digital system designed in this project.}
    \label{fig:digital-system-overview-c5}
\end{figure}


\subsection{Digital signal processing system}

A crucial part in the design of the BCI system is the digital signal processing (DSP) system. The key functions of this system are to digitise, filter and resample the analogue output of the analogue signal processing system presented in Figure \ref{fig:analogue-system-c4}. An overview of the DSP system is shown in Figure \ref{fig:digital-system-c5}.

\begin{figure}[!htb]
    \centering
    \includegraphics[width=0.9\textwidth]{digital-system}
    \caption[Digital signal processing system]{Overview diagram of the core components of the digital signal processing system.}
    \label{fig:digital-system-c5}
\end{figure}

\subsubsection{Sampling and decimation}
The analogue signal $x(t)$ is digitised to $x_1[n]$ using the 12-bit SAR ADC on-board the ESP32. A sampling frequency of $f_s=256$Hz was selected based for several reasons. First, this is a typical value mentioned in the literature as noted in Section \ref{subsection:time-frequency-considerations-c2}. Second, and more importantly, considering the SSVEP stimulus frequency band of 6 - 12Hz as mentioned in Section \ref{subsection:ssvep-stimuli}, and allowing for one reference signal harmonic in CCA-based decoding algorithms, the maximum theoretical frequency required is $f_{\textrm{max}}=24$Hz. In order to satisfy the Nyquist sampling criterion, sampling at $f_s > 2f_{\textrm{max}} = 48$Hz is required. 64Hz was identified as an appropriate fit as it provides some frequency margin for filter roll-off and it is a factor of 256Hz which facilitates decimation by an integer factor.

\begin{figure}
    \centering
    \includegraphics[width=0.85\textwidth]{ssvep-bands}
    \caption[Diagram showing SSVEP stimulus frequency bands and other important frequencies]{Diagram showing SSVEP stimulus frequency bands and other important frequencies. Band (a) represents the fundamental stimulus frequency range and (b) represents the range of first harmonics thereof. The dotted red line is an indicative (idealised) low-pass filter response with corner frequency at $f_c=26$Hz. Including a small safety margin to allow for filter roll-off, $f_n$ represents the Nyquist frequency for this configuration.}
    \label{fig:ssvep-bands}
\end{figure}

Figure \ref{fig:ssvep-bands} shows a high-level view of the sampling and filtering requirements of the system taking into account EEG and SSVEP dynamics, as well as restrictions specific to this project such as memory constraints. 

\subsubsection{Digital filtering}
\label{subsection:digital-filtering}
As indicated in Figure \ref{fig:ssvep-bands}, an ideal low-pass filter for this system would achieve zero pass-band distortion (ripple) between 7 and $24$Hz and steep roll-off after $f_c=26$Hz to achieve complete signal attenuation before the Nyquist frequency $f_n=32$Hz. Obviously, this is not physically realisable and so a trade-off between maximising filter roll-off and minimising pass-band ripple must be sought.  

Infinite impulse response (IIR) filters are typically more suitable for small, resource-constrained DSP systems. Compared to finite impulse response (FIR) filters, they generally offer:
\begin{itemize}
    \item the ability to be implemented recursively
    \item greater computational efficiency
    \item lower memory requirements
    \item improved resolution at lower frequencies
\end{itemize}
Although FIR filters typically offer greater stability and controllability owing to only having zeros in their transfer functions, the aforementioned benefits of IIR filters were deemed to be more important for this application. 

Figure \ref{fig:digital-filt-resp} shows the frequency response of three different commonly used IIR digital filters: type I and II Chebyshev filters, as well an elliptical filter. Each of these filters were designed to meet the following requirements:
\begin{itemize}
\label{list:filter-design-reqs}
    \item maximum filter order of $n=10$
    \item maximum pass-band ripple (below unity gain) of $r_p=0.2$dB 
    \item minimum stop-band attenuation of $r_s=80$dB
\end{itemize}

\begin{figure}[h]
    \centering
    \includegraphics[width=\textwidth]{digital-filt-resp}
    \caption[Frequency response of the digital low-pass filter implemented in firmware on the ESP32]{Frequency response of the digital low-pass filter implemented in firmware on the ESP32. The pre-filtered Nyquist frequency in this implementation is 128Hz: half the 256Hz sampling frequency.}
    \label{fig:digital-filt-resp}
\end{figure}
Observing the magnitude plot in the to half of Figure \ref{fig:digital-filt-resp}, it is evident that the response of the elliptic filter offers the optimal balance of steep roll-off between pass and stop bands and acceptable ripple in the stop and pass bands. This is intuitive: as $r_p$ approaches zero, the elliptical filter becomes a Chebyshev type II filter. As $r_s$ approaches zero, it becomes a Chebyshev type I filter. The phase response of the elliptical filter is also largely linear in the pass-band. For these reasons, a $10^{\textrm{th}}$ order elliptical low-pass filter satisfying the aforementioned design requirements was selected.

\section{Embedded Firmware}
As alluded to in Section \ref{subsection:electronic-hardware-proto}, all firmware for this system was to be developed for the ESP32 SoC developed by Espressif Systems. Espressif offers a comprehensive \href{https://github.com/espressif/esp-idf}{open source} framework called the Espressif IoT Development Framework (ESP-IDF) that can be used to develop ESP32-based applications on Windows, Linux and macOS using C and C++ programming languages. Furthermore, ESP-IDF tooling has been extended to be used with the popular open source platform Arduino. While the ESP-IDF has arguably become the de facto standard for developing complex embedded applications for the ESP32, a different approach was taken in this project for reasons explored below. 

\subsection{MicroPython}
MicroPython is a lean and highly efficient implementation of the Python 3 language that provides a carefully-selected subset of the Python standard library that is optimised for microcontrollers and other resource-constrained targets \cite{micropython}. In fact, MicroPython is a full Python compiler and runtime implemented in C (C99) that runs natively (on bare metal) on several architectures. It is an open source project started by Damien George in 2013 that has since gained great traction from the open source community with constant feature updates and fixes. Some of its notable features include \cite{micropython}: 
\begin{enumerate}
    \item compact enough to fit and run within 256k of ROM (code space) and 16k of RAM.
    \item many different compile-time configuration options
    \item support for many architectures (both for microcontrollers and fully-fledged microprocessors): x86, x86-64, ARM, ARM Thumb, Xtensa
    \item inline assembler for Thumb and Xtensa instruction sets
    \item a cross-compiler that allows ordinary Python scripts to be cross-compiled into bytecode that can be loaded from non-volatile flash instead of RAM.
    \item explicit memory errors (\texttt{MemoryError}) for heap exhaustion
    \item explicit run time errors (\texttt{RuntimeError}) when reaching the stack limit
\end{enumerate}

The technical features of MicroPython listed above have some very useful practical implications for application development on the ESP32 that are not available when developing using the ESP-IDF in C/C++. Practical advantages of MicroPython include:
\begin{itemize}
    \item The convenience of using the Python programming language which offers far more high-level functions and constructs over C or C++.
    \item A real-time interactive Python prompt (REPL) that allows commands to be tested and executed immediately over a serial connection. For some MicroPython ports, including that of the ESP32, a WebREPL is also provided that allows for interactivity over a wireless network connection.
    \item Support for both microcontroller and desktop architectures such as X86 allows MicroPython code to be tested with exact equivalence on a personal computer without having to interact with a microcontroller. This can be very useful for prototyping and algorithm development.
    \item A very familiar and intuitive micro-directory and module structure that emulates regular Python modules (explored more below).
    \item A community of open source developers that can very easily contribute new modules and peripheral functionality 
\end{itemize}

\subsection{Module structure}
As alluded to above, a very convenient feature of MicroPython is its intuitive module/file structure. It allows python modules in the form of \texttt{.py} files to be organised in folders that serve as packages as with regular Python. This file structure is stored directly in flash storage on the device and can be read or written to at any time during development. Consider the module structure of the MicroPython application developed in this project below:

\dirtree{%
.1 /.
.2 boot.py.
.2 main.py.
.2 .env.
.2 lib.
.3 core.py.
.3 decoding.py.
.3 peripherals.py.
.3 \dots.
.2 certificates.
.3 dab0ac2b5c-private.pem.key.
.3 dab0ac2b5c-public.pem.key.
.3 dab0ac2b5c-certificate.pem.crt.
.2 logs.
.3 log-17082021-1.json.
.3 log-10072021.txt.
}
The root directory contains two important files - \texttt{boot.py} and \texttt{main.py} - that are run in sequence at boot time. The \texttt{boot.py} script is typically used for internal system processes and \texttt{main.py} is used for any user functionality that should automatically be run at boot. Also in the root directory is the \texttt{.env} file which can be used to store sensitive environment variables such as Wi-Fi passwords or API key strings. The \texttt{lib} directory (package) contains the modules that implement the actual functionality of the digital system, including peripheral management, decoding and computation and networking (note that not all modules are shown above). Amazon Web Services (AWS) key files and certificates for secure MQTT communication are stored in the \texttt{certificates} folder. Finally, a \texttt{logs} directory shows how log files can also be very conveniently stored in arbitrary formats such as \texttt{.json} or \texttt{.txt} in non-volatile flash memory. Listing \ref{listing:dir-structure} below demonstrates how user-specific modules and files can be imported and used within a new script once flashed to the microcontroller's non-volatile memory. 

\begin{listing}[h]
\small
\begin{minted}{python}
# import user-specific functions from non-volatile memory
from lib.decoding import cca
from lib.computation import solve_gen_eig_prob as solve_eig

# MicroPython emulation of the standard Python `os` module
import os

# list root directory
print(os.listdir('/'))

# open and read a file from flash
with open('/logs/log-10072021.txt') as f:
    print(f.read())
\end{minted}
\caption[Basic MicroPython code to import user-specific modules and read from a text file in non-volatile storage.]{Basic MicroPython code to import user-specific modules and read from a text file in non-volatile storage. Note that the syntax is identical to ordinary Python.}
\label{listing:dir-structure}
\end{listing}

\subsection{Numerical computation}
An additional important motivation for using MicroPython over the more traditional C/C++ approach is the ability to use an extremely convenient open source MicroPython module called \texttt{ulab}\footnote{Source code and documentation available \href{https://github.com/v923z/micropython-ulab}{here}.}. The \texttt{ulab} project was created to offer a subset of the very popular \texttt{numpy} numerical computing library for Python which offers highly performant array computing and linear algebra functionality. It also implements a small subset of functionality offered by \textt{scipy}; the equally popular scientific computing library for Python. Some of the most notable features offered by this module include \cite{ulab}:
\begin{enumerate}
    \item compact, iterable and sliceable array structures for numerical data up to 4 dimensions
    \item extremely frugal with RAM usage
    \item efficient, vectorised computations on multi-dimensional arrays
    \item basic linear algebra functionality such as matrix inversion, multiplication, determinants, Cholesky and QR decomposition
    \item fast Fourier transforms (FFTs)
    \item implementation of digital filters using second order sections (SOS)
    \item basic numerical approximation and function minimisation
\end{enumerate}
Being able to perform matrix and array operations, filtering and numerical approximation on a microcontroller in an elegant, Pythonic way is clearly an invaluable asset for this project. Indeed, for any project of this sort, it is somewhat amazing. Listing \ref{listing:ulab-intro} briefly demonstrates just how effortlessly complex numerical computations can be performed using \texttt{ulab}.

\begin{listing}[h]
\small
\begin{minted}{python}
# import ulab numpy module
from ulab import numpy as np

# create an arbitrary positive definite, symmetric 3x3 matrix
# A can be sliced like A[i0:i1, j0:j1]
A = np.array([[25, 15, -5], [15, 18,  0], [-5,  0, 11]])

# compute lower triangular square root or A using Cholesky decomp
A_sqrt = np.linalg.cholesky(A)

# compute determinant of A
det_A = np.linalg.det(A)

# compute inverse of A
A_inv = np.linalg.inv(A)

# compute Moore-Penrose pseudoinverse of A = (A^T.A)^-1.A^T
A_pinv = np.dot(np.linalg.inv(np.dot(np.transpose(A), A)), np.transpose(A))

# many other utility functions such as argmax(), argsort(), convolve()

\end{minted}
\caption{Illustration of the convenience offered by the \texttt{ulab} module for linear algebra and general numerical computing.}
\label{listing:ulab-intro}
\end{listing}

\subsection{Networking}
Once processed and decoded, data from many BCI devices must be streamed to a central cloud service. The design of this cloud service was beyond the scope of this project. However, it was specified that the Amazon Web Services (AWS) IoT Core service was to be used. This service allows IoT devices to securely exchange small data payloads over the Internet using the MQTT protocol. MQTT is a lightweight, publish-subscribe protocol that has become commonplace in the realm of IoT communication. As mentioned in Section \ref{subsection:electronic-hardware-proto}, the ESP32 SoC has integrated Wi-Fi functionality. MicroPython includes generic socket functionality that can be used to implement an MQTT client for interfacing with AWS.

\begin{figure}[!htb]
    \centering
    \includegraphics[width=\textwidth]{network-diagram}
    \caption{Diagram showing the client-broker MQTT interface for relaying data captured by remote BCI devices to an AWS cloud service and thereafter, to a web application for visualisation. }
    \label{fig:networking-diagram-c5}
\end{figure}

As seen in Figure \ref{fig:networking-diagram-c5}, the MQTT protocol defines a message broker (server) and several clients. In this case, the AWS IoT service acts as an MQTT broker that receives messages published by remote ESP32 BCI clients. Each of the clients have a unique client ID but subscribe to a common topic that the AWS service expects. Although the design of subsequent elements in the AWS cloud pipeline was not included in this project, a schematic representation is presented in Figure \ref{fig:networking-diagram-c5}: data received from BCI clients over MQTT is streamed to a web application that can be view in real time.

\subsection{Logging}
% Discuss logging server and the potential for network logging ot anywhere
Data logging from the ESP32-based BCI hardware was crucial for this project. This enabled the gathering of experimental data, but also various signals and messages for debugging during development. By leveraging the built-in WiFi functionality on the ESP32, a basic wireless logging system was created. 

\begin{figure}[!htb]
    \centering
    \includegraphics[width=0.9\textwidth]{web-logger-diagram}
    \caption{Diagram showing the client-broker MQTT interface for relaying data captured by remote BCI devices to an AWS cloud service and thereafter, to a web application for visualisation. }
    \label{fig:web-logger-c5}
\end{figure}

Figure \ref{fig:web-logger-c5} provides an overview of the web-based logging system. Potentially many ESP32 devices collect data and periodically log data to a web server. Typically, a REST API (application progamming interface) is deployed on the server which accepts requests from clients and processes and stores request data. In this project, a basic REST API\footnote{source code available \href{https://github.com/JamesTev/EEG-decoding/blob/master/eeg_lib/logging_server.py}{here}.} was created using \href{https://flask.palletsprojects.com/en/2.0.x/}{Flask}, the open source web micro-framework written in Python. 

\begin{listing}[h]
\small
\begin{minted}{python}
{
    "client_id": "esp1", # used to identify which device data was received from
    "timestamp": 1630235886, # used for synchronisation and periodicity monitoring
    "trial_id": "test_123", # used to group trials 
    "decoded_data": { # dict of stim. freqs and their decoded outputs
        "7": 0.12,
        "10": 0.56, 
        "12": 0.04
    },
    "raw_data": [[ 0.02859, 0.054, ..., 0.345], # raw data for further offline analysis
                [ 0.412, 0.001, ..., 0.345]]
}

\end{minted}
\caption{Example JSON data payload sent from a remote ESP32 device via \texttt{POST} HTTP request to the logging server}
\label{listing:logging-payload}
\end{listing}
The JSON object in Listing \ref{listing:logging-payload} shows an example of a data payload sent via an ESP32 client to the logging API. Decoded data, as well as raw data used to perform on-board decoding, is compiled into each data payload. Each payload is then sent via an HTTP \textt{POST} request to the logging server. Once received and verified by the API, data is written to a log file stored on the server. Note that in this project, the server and API were run on a local network and data was thus stored on a personal computer. However, the system would work identically if the server was deployed in the cloud. This would allow data to be logged by remote BCI devices from anywhere in the world and would not require users of the devices to run their own local server. This could facilitate a decentralised data acquisition system that could enable wide-scale data collection.  

\subsection{Digital filtering}
The digital low-pass filter described in Section \ref{subsection:digital-filtering} whose frequency response is shown in Figure \ref{fig:digital-filt-resp} was designed using \href{https://docs.scipy.org/doc/scipy/index.html}{SciPy}, the popular open source scientific computing library for Python. In particular, given the filter design requirements, the \texttt{scipy.signal.ellip()} function was used to generate filter coefficients. Using the \texttt{output==`sos'} argument, the filter coefficients were computed in a cascaded second-order sections (SOS) representation. The SOS representation of an IIR filter with digital transfer function $H(z)$ is as follows:
\begin{equation}
H(z)=K\prod_{n=0}^{N-1} \frac{b_{n,\, 0}+b_{n,\, 1} z^{-1}+b_{n,\, 2} z^{-2}}{1+a_{n,\, 1} z^{-1}+a_{n,\, 2}_{z}^{-2}},
\label{eq:sos-structure}
\end{equation}
where $K$ is a gain scalar, $N$ is the number of sections needed to realise the desired filter order and $a_n$ and $b_n$ are the reverse and forward coefficients for section $n$ respectively. Notice that in the form in (\ref{eq:sos-structure}), only a highest order of 2 is present for any component. This cascaded lower-order form offers greater numerical stability and less sensitivity to quantisation of coefficients than standard higher order polynomials.

The filter coefficients, once computed offline, were stored as constants in the the ESP32 firmware (in particular, in the \texttt{lib.signal} module). As there was no need to recompute or otherwise modify these coefficients, they were stored in read-only flash memory to persist across power cycles and save space in RAM. In order to run digital filtering on the ESP32 itself, the convenient \texttt{scipy.signal.sosfilt()} from the \texttt{ulab} module was used. This function simply takes the pre-computed SOS coefficients and the input signal and returns the filtered version. Tests were performed to ensure equivalence between this implementation and the standard implementation in the fully-fledged SciPy module.  

\section{Algorithm Implementation}
\subsection{Eigenvalue algorithms}
As elaborated on in Chapter \ref{chapter:theory-development}, the decoding algorithms explored in this project are all optimisation problems that can be reformulated as eigenvalue problems. Consequently, an `eigen' function capable of finding the eigenvalues (and corresponding eigenvectors) of an arbitrary matrix was crucial in the design of this system. The \texttt{ulab} module conveniently includes a basic eigen function of this sort: \texttt{numpy.linalg.eig()}. However, the severe limitation with this implementation is that it only works with \textit{symmetric} matrices (i.e. matrices with real eigenvalues). For signal decoding, many of the eigenvalue problems arising in this project involve asymmetric matrices. Thus, an alternative solution needed to be sought. 

After much experimentation and difficulty finding an analytical solution that could work\footnote{produce the correct result and maintain numerical stability} under all conditions, it was decided that \textit{iterative} eigenvalue algorithms should be explored. 

\subsubsection{Power Iteration}
The simple but elegant power iteration algorithm explored in Section \ref{subsection:power-iteration-c3} can easily be implemented in firmware. The pseudocode for this algorithm is provided in Algorithm \ref{alg:power-iteration}. The corresponding MicroPython implementation is documented in Listing \ref{app-listing:power-iteration-mpy}.
\begin{algorithm}
\begin{algorithmic}
\State Pick an initial vector $\vv^{(0)}$ with $\|\vv^{(0)}\|=1$ \Comment{Initial unit vector assigned randomly}
\For{$k=1, 2, \dots, \, K$ } \Comment{repeat for $K$ iterations}
    \State $\vw \gets \vw = \mA\vv^{(k-1)}$ \Comment{update eigenvector estimate}
    \State $\vv^{(k)} \gets \vw / \| \vw\|$ \Comment{normalise new estimate}
\EndFor
\end{algorithmic}
\caption{Power iteration}
\label{alg:power-iteration}
\end{algorithm}

\subsubsection{Simultaneous Iteration}
As discussed in Section \ref{subsection:simultaneous-iteration}, the Simultaneous Iteration algorithm provides an effective extension of Power Iteration by solving for all eigenvectors simultaneously. The pseudocode for the implementation of this algorithm is given in Algorithm \ref{alg:simultaneous-iteration}.
\begin{algorithm}
\begin{algorithmic}
\State Pick an initial basis for $\{\vv^{(0)}, \dots, \vv^{(d)} \}$ for $\R^d$
\State Construct the matrix $\mV = \left[\begin{array}{l|l|l}\vv_1^{(0)} & \dots & \vv_d^{(0)}\end{array}\right]$
\State Obtain the factors $\mQ^{(0)}\mR^{(0)} = \mV^{(0)}$ \Comment{compute QR factorisation of $\mV^{(0)}$}
% \Require 
\For{$k=1, 2, \dots, \, K$ }
    \State $\mW \gets \mW = \mA\mQ^{(k-1)}$
    \State Obtain the factors $\mQ^{(k)}\mR^{(k)} = \mW$ \Comment{extract orthonormal column vectors of $\mW$ from $\mQ^{(k)}$}
\EndFor
\end{algorithmic}
\caption{Simultaneous iteration}
\label{alg:simultaneous-iteration}
\end{algorithm}

\subsubsection{QR Iteration}
Finally, as revealed in Section \ref{subsection:qr-iteration}, the QR Iteration algorithm is a more convenient form of the Simultaneous Iteration algorithm that is more commonly used in practice. Pseudocode for its implementation in this project is given in Algorithm \ref{alg:qr-iteration}.
\begin{algorithm}
\begin{algorithmic}
\State $\mA^{(0)} \gets \mA$
\For{$k=1, 2, \dots, \, K$ }
    \State Obtain the factors $\mQ^{(k)}\mR^{(k)} = \mA^{(k-1)}$ \Comment{perform QR factorisation on $\mA^{(k-1)}$}
    \State $\mA^{(k)} \gets \mR^{(k)}\mQ^{(k)}$ \Comment{update step detailed in \cite{panju-iterative-eig}}
\EndFor
\end{algorithmic}
\caption{QR iteration}
\label{alg:qr-iteration}
\end{algorithm}
The MicroPython implementation of this algorithm is presented in Listing \ref{app-listing:qr-iteration-mpy} in Appendix \ref{app:firmware}.

\subsubsection{Generalised eigenvalue algorithm}
The generalised eigenvalue problem for two symmetric matrices $\mA, \, \mB \in \R^{d \times d}$ is given by the form
\begin{equation}
    \mA\vx_{i}=\lambda_{i} \mB \vx_{i}, \quad \forall i \in\{1, \ldots, d\},
    \label{eq:gen-eig-problem}
\end{equation}
where $\vx_i$ and $\lambda_i$ are the eigenvectors and eigenvalues of the system respectively. This problem is relevant to several decoding algorithms explored in this project, including the GCCA and MsetCCA algorithms explored in Chapter \ref{chapter:theory-development}. The proof for solving for $\vx_i$ and $\lambda_i$ is involved but well known and is given in \cite{ghojogh-gen-eig-prob}. However, understanding the algorithm to solve this problem is crucial for enabling implementation in firmware. The generalised eigenvalue algorithm, extracted from \cite{ghojogh-gen-eig-prob}, is given in Algorithm \ref{alg:gen-eig-algo} below.

\begin{algorithm}
\begin{algorithmic}
\State $\Phi_{\mB}, \Lambda_{\mB} \gets \mB \Phi_{\mB}=\Phi_{\mB} \Lambda_{\mB}$ \Comment{compute eigenvectors $\Phi_{\mB}$ and eigenvalues $\Lambda_{\mB}$ of $\mB$}
\State $\widetilde{\Phi}_{\mB} \gets \widetilde{\Phi}_{\mB}=\Phi_{\mB} \Lambda_{\mB}^{-1 / 2} \approx \Phi_{\mB}\left(\Lambda_{\mB}^{1 / 2}+\varepsilon I\right)^{-1}$ \Comment{multiply above by $\Lambda_{\mB}^{-1 / 2}$, define intermediate $\widetilde{\Phi}_{\mB}$}
\State $\widetilde{\mA} \gets \widetilde{\mA}=\widetilde{\Phi}_{\mB}^{\top} \mA \widetilde{\Phi}_{\mB}$ \Comment{define intermediate $\widetilde{\mA}$}
\State $\Phi_{\mA}, \Lambda_{\mA} \gets \widetilde{\mA} \Phi_{\mA}=\Phi_{\mA} \Lambda_{\mA}$ \Comment{compute eigenvectors $\Phi_{\mA}$ and eigenvalues $\Lambda_{\mA}$  of $\widetilde{\mA}$}
\State $\Lambda \gets \Lambda=\Lambda_{\mA}$ \Comment{notice that eigenvalues of $\widetilde{\mA}$ are the generalised eigenvalues }
\State $\Phi \gets \Phi=\widetilde{\Phi}_{\mB} \Phi_{\mA}$ \Comment{compute generalised eigenvectors}
\State return $\Phi$, $\Lambda$ \Comment{return the generalised eigenvector and eigenvalue matrices}
\end{algorithmic}
\caption{Generalised eigenvalue algorithm}
\label{alg:gen-eig-algo}
\end{algorithm}

The MicroPython implementation of the generalised eigenvalue algorithm can be found in Listing \ref{app-listing:gen-eig-prob-mpy} in Appendix \ref{app:firmware}.

\subsection{Decoding algorithms}
\subsubsection{CCA}
\subsubsection{GCCA}
\subsubsection{MsetCCA}




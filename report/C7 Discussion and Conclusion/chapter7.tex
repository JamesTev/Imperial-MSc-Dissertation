%% Chapter 7: Discussion and Conclusion
\chapter{Discussion and Conclusion}
\label{chapter:discussion-conclusion}

\graphicspath{ {report/Chapter7/assets/} } 

\section{Discussion}
Lessons learnt

\subsection{Challenges encountered}
Discuss numerical challenges, precision issues, memory constraints
Discuss some of the auxiliary computational approaches used such as eigenvalue solvers etc

Mention how most other reports like MsetCCA and GCCA compare effect of following on acc:
- number of harmonics 
- number of channels
obv not possible in this project.

Mention how open source tools been used exclusively and all components of this proj were developed in Jupyter notebooks - incl. filtering, DAQ, preprocessing, decoding, networking etc. This could be a great platform for young people to learn more. Mention also how micropython lends itself to this; you can use the interactive notebook with the ESP32 over serial which is much more intuitive and user friendly than having to re-flash and test.

Mention somewhere that the networking aspect with MQTT can stream data up to 5Hz but was only required to do so at around 1Hz (2s sampling window with 50\% overlap)



\section{Conclusion and Future Work}
Evaluate if objectives initially laid out were achieved
- if not, why not
- if they had been stated differently/more realistically, would the outcome have been different?
- what would need to be done differently next time

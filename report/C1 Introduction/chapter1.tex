\chapter{Introduction}
\label{chapter:introduction}

\graphicspath{ {report/Chapter1/assets/} } 

\section{Background and Motivation}
\label{secetion:background-motivation}

A brain-computer interface (BCI), also known as brain-machine interface (BMI), is a system that allows computers or other external devices to be controlled by electroencephalographic (EEG) activity alone \cite{bci-survey-nicolas-alonso}. The most common application of BCIs is to provide a means of communication to people who have lost the means to do so in a normal capacity, typically due to injury, trauma or preexisting impairments of the neuromuscular system. Particularly in the case of fully paralysed or 'locked in' patients suffering disorders such as amyotrophic lateral sclerosis (ALS), strokes or spinal cord injuries, advances in BCI technology pose a promising means of restoring the ability to communicate with the outside world \cite{bci-wolpaw}. This is made possible by measuring and characterising (decoding) brain activity in order to translate raw EEG signals into external control commands without any neuromuscular interaction. External control targets may be in the form of computer-based spellers, speech synthesizers or neural prostheses \cite{bci-survey-nicolas-alonso}. 

In some cases of severe impairment, invasive EEG treatment is required which involves implantation of electrode arrays on the surface of the brain. However, non-invasive EEG is far more prevalent as it only requires surface electrodes placed on the scalp of a subject and is thus safer, more convenient and far more accessible \cite{duart-comparing-ssvep-stimuli}. BCIs using this technology have found applications in smart home appliances, disability assistance and human-computer interaction \cite{zhao-stimulus-layout-effect}. For example, a patient bound to their bed due to neuromuscular impairments could interact with a BCI to adjust lighting, heating, television or other appliances. 

A challenge with existing BCI technologies is their cost which, in many cases, prohibits their use on a mass scale. In this project, a novel, ultra low-cost\footnote{In the region of 10 to 20 times less expensive than the cheapest comparable commercial model} EEG-based BCI system will be created in order to investigate the viability of acquiring and decoding EEG activity from an audience of 100 people simultaneously. 

Importantly, the success of this project would provide great opportunities for public engagement and education in the domain of EEG sensing and other neurotechnologies. For example, it is conceivable that a future version of this prototype could be used in summer schools or other educational activities to give young learners hands-on experience with this otherwise expensive and largely inaccessible technology. More widespread public demonstrations such as these will increase and improve advocacy for this technology which may, at present, be somewhat stigmatised in some communities due to its opaqueness and potentially, due to the negative association with mind control or other dramatised pop culture analogues.

Increased public interest and accessibility, in turn, will likely encourage further funding for this fascinating field of research. Furthermore, while the initial demonstration of this project will be performed on healthy individuals, it could provide great promise for facilitating large scale, inclusive interaction among people with neuromuscular impairments such as those mentioned above.  

\section{Objectives of the Study}
The core focus of this study is to develop real time decoding and communication of raw analogue EEG signals acquired from a proprietary EEG hardware device developed by the Next Generation Neural Interfaces (NGNI) Lab at Imperial College London. These devices will be worn by up to 100 people simultaneously and will be used to facilitate collaborative control in a multiplayer game. Consequently, decoding, communication and visualisation needs to happen concurrently and in real time between devices. 

Specific objectives of this project are outlined below:
\begin{enumerate}
    \item Review the literature pertaining to EEG signal acquisition and decoding. Perform a brief survey of common EEG paradigms and decide on a particular paradigm for further investigation based on a clear rationale. 
    \item Review decoding techniques and algorithms for the selected EEG paradigm. Record the most effective algorithms as specified in the literature, as well as their analytical derivations in preparation for implementation. 
    \item Scope firmware most suitable for the microcontroller used in the EEG hardware supplied by the NGNI Lab.
    \item Design and implement embedded software required to sample EEG signals and perform online decoding thereof using previously identified decoding algorithms.
    \item Determine the optimal network protocol and structure to communicate decoded data and any other pertinent device-specific information to a cloud service.
    \item Devise testing mechanisms to verify the validity of obtained results, including significance tests to demonstrate non-trivial decoding skill.
    \item Critically evaluate the performance of the system, both with reference to the objectives of this study and the performance of the most comparable commercially-available system.
    \item Provide a discussion of results obtained and provide reasons for any shortcomings or anomalies. Provide suggestions for further improvement.
    
\end{enumerate}


\subsection{Research questions to be investigated}
This project aims to address the following core research questions: 
\begin{itemize}
    \item Can an ultra low-cost (in the region of £20) device be used to \textit{reliably} decode EEG signals of interest in real time? For a minimum case of reliability, there should be quantifiable measure of significance that demonstrates that events/signals of interest can be decoded beyond just pure chance. 
    \item If so, can such a device perform all computation necessary for signal decoding on-device (without the need to outsource computation to the cloud or end client, for example)? 
\end{itemize}
\subsection{Significance of this work}
If able to fulfil its aforementioned objectives, this device would serve as a significant proof-of-concept system in the area of EEG signal analysis. Importantly, as specified by the constraints below, this device is to be built using low cost, easily accessible components. Commercial EEG devices are typically priced from several hundred GBP and upwards [REF NEEDED]. Very few, if any, devices are available for less than £100. Therefore, while likely slightly less capable than the aforementioned commercial counterparts, the device investigated in this project could pose an interesting low-cost option for large scale EEG data acquisition, testing and experimentation. Furthermore, as alluded to in the motivation in Section \ref{secetion:background-motivation}, it could become a valuable pedagogical tool for engaging and educating young learners about basic neurophysiology and neurotechnology; fields that would otherwise be difficult to access below a tertiary education level.

% ** Provide a reference that demonstrates the demand for and/or usefulness of these devices. Need to show why having a device like this would actually be significant.

\section{Scope and Constraints}
\label{section:scope-constraints-intro}
The task of using this device at scale in a public setting imposes some unique practical constraints. These are outlined below:
\begin{itemize}
    \item production at scale: 100+ replicas of the BCI device need to be produced.
    \item very tight budget of $\approx$ £20 per device.
    \item hardware must be self-contained and so all computation should happen on the hardware if possible. 
    \item there is to be minimal calibration required from the user. If any, it should be done automatically and should not detract significantly from the overall user experience.
    \item real time decoding and feedback of signals (or as close to real time as possible).
    \item device should be non-invasive and cannot use `wet' electrodes that would hamper the user experience.
\end{itemize}
The EEG hardware provided by the NGNI Lab is based on the low-cost Espressif ESP32 SoC based on the Tensilica Xtensa LX6 MCU. All firmware must be developed on this device.

Finally, every effort must be taken to use exclusively \textbf{open source} software, tools and other resources in developing this system. This will create a platform most suitable for the education space as it will afford the ability to easily modify, extend or deconstruct the system as a result of having accessible resources; both on a physical hardware/firmware level, as well as in the form of vast open source communities and documentation.

\subsection{Constraint implications}
The constraints listed above have substantial implications on my project. One of the largest constraints is the budget of £20; entry level consumer-grade BCIs (brain computer interfaces) typically cost in the region of several hundred USD. As a result, the device in this project will only have \textbf{two active electrodes} measured in a differential configuration to yield only a \textbf{single active channel}. Although most BCIs have many more active channels, many studies have shown that a viable decoding system can be achieved with only a limited number of channels \cite{Wang2011}, some with even as few as 2 \cite{acampora-dataset}. 

The prohibition of wet electrodes is also significant. Wet electrodes are known to significantly improve the quality\footnote{by reducing contact impedance} of the electrical connection to the scalp which would invariably improve SNR. 
\section{Plan of Development}
This report begins with a review of the basics of neurophysiology and how electrical signals are used by the human brain. An overview of the mechanics of electroencephalography (EEG) is presented, followed by some of the popular EEG paradigms explored in the literature. A particular paradigm is selected for further analysis. The current state-of-the-art in EEG signal decoding for the selected EEG paradigm is explored both from a technological and theoretical perspective. 

Chapter \ref{chapter:theory-development} expands on the theory of decoding algorithms that are most applicable to this project. This theory expansion is vital as it lends a far easier understanding of the final implementation in software. Various computational approaches are explored and their advantages and disadvantages analysed. 

Chapter \ref{chapter:system-design} explores the design of the embedded software required to execute the decoding algorithms introduced in the previous chapter. Auxiliary components, such as networking and communication infrastructure, are also explored. Computational challenges and the evolution of their solutions are presented. 

Results of the implemented decoding system are provided in Chapter \ref{chapter:results}. Results obtained in various scenarios are documented, including decoding performance using data gathered using a low level commercially available OpenBCI headset, as well as data using the final hardware prototype created as part of this project. 

Finally, a discussion of the results obtained is provided in Chapter \ref{chapter:discussion-conclusion}. Conclusions are drawn and suggestions for improvement in further work are given. 